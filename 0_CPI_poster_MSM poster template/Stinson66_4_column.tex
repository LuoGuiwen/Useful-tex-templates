\documentclass[final]{beamer}

%\usepackage[scale=1.24]{beamerposter} % Use the beamerposter package for laying out the poster

\usepackage[orientation=portrait,size=a0,scale=1.6,debug]{beamerposter}                       

\usetheme{confposter} % Use the confposter theme supplied with this template

\setbeamercolor{block title}{fg=ngreen,bg=white} % Colors of the block titles
\setbeamercolor{block body}{fg=black,bg=white} % Colors of the body of blocks
\setbeamercolor{block alerted title}{fg=white,bg=dblue!70} % Colors of the highlighted block titles
\setbeamercolor{block alerted body}{fg=black,bg=dblue!10} % Colors of the body of highlighted blocks
% Many more colors are available for use in beamerthemeconfposter.sty

%-----------------------------------------------------------
% Define the column widths and overall poster size
% To set effective sepwid, onecolwid and twocolwid values, first choose how many columns you want and how much separation you want between columns
% In this template, the separation width chosen is 0.024 of the paper width and a 4-column layout
% onecolwid should therefore be (1-(# of columns+1)*sepwid)/# of columns e.g. (1-(4+1)*0.024)/4 = 0.22
% Set twocolwid to be (2*onecolwid)+sepwid = 0.464
% Set threecolwid to be (3*onecolwid)+2*sepwid = 0.708

\newlength{\sepwid}
\newlength{\onecolwid}
\newlength{\twocolwid}
\newlength{\threecolwid}
\setlength{\paperwidth}{48in} % A0 width: 46.8in
\setlength{\paperheight}{36in} % A0 height: 33.1in
\setlength{\sepwid}{0.005\paperwidth} % Separation width (white space) between columns
\setlength{\onecolwid}{0.25\paperwidth} % Width of one column
%\setlength{\twocolwid}{0.464\paperwidth} % Width of two columns
%\setlength{\threecolwid}{0.708\paperwidth} % Width of three columns
\setlength{\topmargin}{-0.6in} % Reduce the top margin size
%\usepackage[bottom=0.5in]{geometry} 
%-----------------------------------------------------------

\usepackage{graphicx}  % Required for including images

\usepackage{booktabs} % Top and bottom rules for tables
\usepackage{amsmath}

\usepackage{tikz}
\usepackage{wrapfig}
\usetikzlibrary{decorations.markings}
\usetikzlibrary{shapes.geometric}
\usepackage{pgfplots}
\usetikzlibrary{shapes,arrows,automata}
\usetikzlibrary{calc}
\usetikzlibrary{shapes,arrows}
\usetikzlibrary{fit}
\usepackage{graphicx}

\newcommand{\mbu}{{\bf{m}}}
\newcommand{\pbu}{{\bf{p}}}
\newcommand{\qbu}{{\bf{q}}}

\newcommand{\sbu}{{\bf{s}}}
\newcommand{\wbu}{{\bf{w}}}
\newcommand{\xbu}{{\bf{x}}}
\newcommand{\ybu}{{\bf{y}}}
\newcommand{\zbu}{{\bf{z}}}

\newcommand{\nhu}{\mathcal{N}}
\newcommand{\uu}{\mathcal{U}}
\newcommand{\cu}{\mathcal{C}}
\newcommand{\du}{\mathcal{D}}
\newcommand{\lu}{\mathcal{L}}
\newcommand{\F}{{\mathbb F}}
\newcommand{\abu}{{\bf{a}}}
\newcommand{\abuj}{{\bf{a_j}}}
\newcommand{\bbu}{{\bf{b}}}
\newcommand{\vbu}{{\bf{v}}}
\newcommand{\ubu}{{\bf{u}}}
\newcommand{\dbu}{{\bf{d}}}
\newcommand{\tbu}{{\bf{t}}}
\newcommand{\cbu}{{\bf{c}}}
\newcommand{\kbu}{{\bf{k}}}
\newcommand{\hbu}{{\bf{h}}}
%\newcommand{\qbu}{{\bf{q}}}

\newcommand{\sstm}{\textbf{Share}_m^t}
\newcommand{\rectm}{\textbf{Rec}_m^t}
\newcommand{\comtm}{\textbf{Comm}^t}
\newcommand{\vrfytm}{\textbf{SVerify}^t}

\newcommand{\sstt}{\textbf{Share}_3^2}
\newcommand{\rectt}{\textbf{Rec}_3^2}

\newcommand{\comtt}{\textbf{Comm}^2}
\newcommand{\vrfytt}{\textbf{SVerify}^2}


\newcommand{\siml}{\texttt{Sim}}
\newcommand{\view}{\texttt{VIEW}}

%\newcommand{\C}{\mathcal{C}}
\newcommand{\su}{\mathcal{S}}
\newcommand{\E}{\mathcal{E}}
\newcommand{\D}{\mathcal{D}}
\newcommand{\K}{\mathcal{K}}

% \newcommand{\ebu}{\textbf{E}}
% \newcommand{\Dbu}{\textbf{D}}
\newcommand{\ebu}{\mathcal{E}}
\newcommand{\Dbu}{\mathcal{D}}
\newcommand{\A}{\mathcal{A}}
\newcommand{\autenc}{\mathcal{AE}}

\newcommand{\sig}{\textsf{Sig}}
\newcommand{\vrfy}{\textsf{Vrfy}}

\newcommand{\sk}{\textsf{sk}}
\newcommand{\pk}{\textsf{pk}}
\newcommand{\idealfnc}{{\mathcal F}}



%----------------------------------------------------------------------------------------
%	TITLE SECTION 
%----------------------------------------------------------------------------------------

\title{Speeding Up Multi-Scalar Multiplication Towards Efficient zkSNARKs} % Poster title

\author{Guiwen Luo, Guang Gong} % Author(s)
\institute{\{g27luo,\ ggong\}@uwaterloo.ca\\ Department of Electrical and Computer Engineering, University of Waterloo, Waterloo, ON, N2L3G1, CANADA} % Institution(s)

%----------------------------------------------------------------------------------------
\begin{document}

\addtobeamertemplate{block end}{}{\vspace*{2ex}} % White space under blocks
\addtobeamertemplate{block alerted end}{}{\vspace*{2ex}} % White space under highlighted (alert) blocks

\setlength{\belowcaptionskip}{2ex} % White space under figures
\setlength\belowdisplayshortskip{2ex} % White space under equations

\begin{frame}[t] % The whole poster is enclosed in one beamer frame
\begin{columns}[t] % The whole poster consists of three major columns, the second of which is split into two columns twice - the [t] option aligns each column's content to the top
\begin{column}{\sepwid}\end{column} % Empty spacer column

% The first column
\begin{column}{\onecolwid} 

\setbeamercolor{block title}{fg=blue,bg=white} % Change the block title color
\begin{block}{Problem} 
\small
    Multi-scalar Multiplication (MSM) over fixed points: 
    	\begin{equation*}\label{eq_multi_scalar_multiplication}
				S_{n,r} = a_1P_1+a_2P_2+...+a_nP_n,\ 0\le a_i< r, P_i\in E.				
		\end{equation*}
How can we compute it efficiently for large $n: 2^{16}\le n \le 2^{20}$?
\end{block}

\begin{block}{Motivation} 
\small
\begin{itemize}
	\item MSM dominates the time consumption in zero-knowledge succinct non-interactive argument of knowledge (zkSNARK) schemes with  pairing-based trusted setup.		
			
	\item 	 Circuit size in Zcash: for single hash, SHA-256, the number of multiplication gates is about 32 thousands; for nested hash, several millions.

\end{itemize}

%\vspace{0.25in}

\begin{center}
		\begin{figure}
			\centering
			\includegraphics[width=0.4\linewidth]{Figures/Zcash_logo}
			\caption{\centering credit: \url{https://en.wikipedia.org/wiki/Zcash}}
		\end{figure}
\end{center}					
\end{block}


\begin{block}{Pippenger's bucket method example}
\begin{small}
	\begin{itemize}	
		\item Example:
		\begin{eqnarray*}
		S_{13,8}=2 P_{1}+3 P_{2}+7 P_{3}+6 P_{4}+5 P_{5}+1 P_{6}+3 P_{7}\\
		+6 P_{8}+2 P_{9}+7 P_{10}+1 P_{11}+4 P_{12}+5 P_{13}.
		\end{eqnarray*}
		
		\item All points are sorted into 7 buckets according to the scalars $\{1, \cdots, 7\}$:
		\begin{equation*}
		\begin{aligned}
		S_{13,8}=& 1 \cdot\left(P_{6}+P_{11}\right)+2 \cdot\left(P_{1}+P_{9}\right)+3 \cdot\left(P_{2}+P_{7}\right)+4 \cdot P_{12} \\
		&+5 \cdot\left(P_{5}+P_{13}\right)+6 \cdot\left(P_{4}+P_{8}\right)+7 \cdot\left(P_{3}+P_{10}\right) \\
		:=& 1 S_{1}+2 S_{2}+\ldots+7 S_{7}.\\
		\end{aligned}
		\end{equation*}					
		\item {The accumulated sum $\sum_{i=1}^{7}iS_i$ can be computed via
		\begin{equation*}
		\begin{aligned}
		&\ \ \ \ \  S_7 \\
		&+ (S_7+S_6)\\
		&+ (S_7+S_6+S_5)\\
		&\cdots \\
		&+(S_7+S_6+S_5+\cdots +S_1).
		\end{aligned}
		\end{equation*}	}			
		
		\item $\{S_i\}$: $13 - 7 = 6$ additions,\\
		$\sum_{i = 1}^{7} iS_i$: $2\times 6 = 12$ additions. \\
		In total, 18 additions.	
	\end{itemize}
\end{small}
\end{block}


\end{column} 
\begin{column}{\sepwid}\end{column} % Empty spacer column

% The second column
\begin{column}{\onecolwid} 

\begin{block}{Pippenger's bucket method}
\small
	\begin{itemize}	
	\item If $r$ is small enough:
	\begin{eqnarray*}
	S_{n,r}= a_1P_1+a_2P_2+\cdots + a_nP_n.
	\end{eqnarray*}
	
	\item All points are sorted into $r-1$ buckets according to the scalars,
	\begin{equation*}
	\begin{aligned}
	S_{n,r}=& 1 S_{1}+2 S_{2}+\ldots+(r-1) S_{r-1}\\
	=& S_{r-1} + (S_{r-1}+S_{r-2})+\cdots\\
	&+(S_{r-1}+S_{r-2}+\cdots +S_1).
	\end{aligned}
	\end{equation*}
	
	\item $S_i$'s: $n - r + 1$ additions,\\
	$\sum_{i = 1}^{r-1} iS_i$: $2\times (r-2)$ additions.\\
	In total, $n+r -3$ additions.	
	\end{itemize}
			
	\begin{itemize}	
			\item If $r$ is big (over BLS12-381 curve, $r\approx 2^{256}$), every scalar is decomposed into its $q$-ary form,
			\begin{eqnarray*}
			a_i &=& a_{i0} + a_{i1}q +\cdots+ a_{i,h-1}q^{h-1} \\
			S_{n,r}&=& a_1P_1+a_2P_2+\cdots + a_nP_n\\
			&=&\sum _{i=1}^n\sum_{j=0}^{h-1}a_{ij}\cdot (q^{j}P_i), 0 \le a_{ij}< q,\\
			& :=& S_{nh,q}.
			\end{eqnarray*}
			
			\item Precomputation ($nh$ Points):
			\begin{equation*}
			\{q^{j}P_i\ |\ i= 1,2,...,n,\ j=0,1,2,...,h-1\}. 
			\end{equation*}
			
			\item Using aforementioned method, all points are sorted into $q-1$ buckets,
			in total, $nh+ q -3$ additions.
	\end{itemize}
\end{block}

\begin{block}{General framework}
\small
	\begin{itemize}	
		\item Let us summary the framework of computing MSM,
		$$ S_{n,r} = S_{hn,q}=\sum_{i=1}^{n}\sum_{j=0}^{h-1} a_{ij}q^j P_i, 0\le a_i\le q $$
		
		\item If $a_{ij} = m_{ij}b_{ij}$, $m_{ij}\in M$ (multiplier), $b_{ij}\in B$ (bucket),
			\begin{equation*}
			\begin{aligned}
			S_{hn,q}&= \sum_{i=1}^{n}\sum_{j=0}^{h-1}b_{ij}\cdot (m_{ij}q^jP_i).\\
			\end{aligned} 
			\end{equation*}
		\item Precompute ($nh|M|$ points) \\
		$\{mq^jP_i\ |\ 1\le i\le n,0\le j\le h-1,m\in M\}$,\\
		then it takes $\approx nh +|B|$ additions to compute $S_{n,r}$.
		
		\only {\item Pippenger's bucket method,
		$M =\{1\},\ B =\{0,1,2,...,q-1\}$, it takes $\approx nh + q$ additions.}
		\item Pippenger's bucket method variant (notice that $-P$ can be easily computed given $P$),\\
		$M =\{1,-1\},\ B =\{0,1,2,...,\lceil q/2\rceil\}$, it takes $\approx nh + q/2$ additions.
	\end{itemize}

\end{block}
\end{column} 


\begin{column}{\sepwid}\end{column} % Empty spacer column

% The third column
\begin{column}{\onecolwid} 

\setbeamercolor{block title}{fg=red,bg=white} % Change the 
\begin{block}{New alogorithm}
\small
	\begin{center}	
		\textbf{Question:} Construct $B$, \textit{s.t.} $|B|\approx q/\ell$ ?\\ then $S_{n,r}$ takes $\approx hn + q/\ell$ additions.
	\end{center}

		\begin{itemize}	
			\item (G.W. Luo, G. Gong, C.K. Weng) Let $q$ be a prime \textit{s.t.} \\
			$2$ is a primitive element in $\mathbb{F}_q$,\\
	 $\ell$ and $h$ be small positive integers \textit{s.t.}\\ $2^{\ell-1}<q$ and $q^{h-1} < r \le 2^{\ell-1} q^{h-1}$.
			\item The multiplier set is 
			\begin{equation*}\label{eq_multiplier_m_new}
			M =\{2^i\ |\ 0\le i\le \ell - 1\} \cup \{-1\},
			\end{equation*}
			\item The corresponding bucket set ($|B|\approx q/\ell$) is
			\begin{equation*}\label{eq_bucket_b_new}
			B =\{i\ | \ 0\le i\le 2^{\ell-1} \} \cup
			\{ 2^{i\cdot\ell}\bmod q\ |\ 0\le i \le \lfloor q/\ell \rfloor \}.
			\end{equation*}
			The idea behind the construction is that 
			\begin{equation*}
			\begin{aligned}
			\{i\ |1 \le i < q \} &= \{2^i \bmod q\ | \ 0\le i \le q-2\}.
			\end{aligned}
			\end{equation*}	
			
			\item Note: Elements in the bucket set is no longer consecutive, a new accumulation algorithm is needed.		
		\end{itemize}
\end{block}

\begin{block}{Further optimization}
\small
\begin{itemize}	
		\item	Observation: 										
		\begin{equation*}
				\begin{aligned}
				\{i\ |1 \le i < q \} &= \{2^i \bmod q\ | \ 0\le i \le q-2\}\\
				& = \{2^i \bmod q\ | \ -(q-3)/2\le i \le (q-1)/2\}.
				\end{aligned}
		\end{equation*}	

 Bucket set can be further reduced to ($|B|\approx q/(2\ell)$)
			\begin{equation*}
			B=\left\{i \mid 0 \leq i \leq 2^{\ell}\right\} \cup\left\{2^{i \cdot \ell} \bmod q \mid 0 \leq i \leq\lfloor(q-1) / 2 \ell\rfloor\right\}
			\end{equation*}
			
		\item GLV endomorphism 
		\[\lambda P = \lambda\cdot (x,y) = (\xi x, y),\  \lambda^3=1\in \mathbb{F}_r, \xi^3=1 \in \mathbb{F}_p,\]			
		$\lambda \approx \sqrt{r}$, every scalar 
		$a = a_0 + a_1\lambda,$	so			
		\[S_{n,r} = S_{2n,\lambda}.\]	
		It further reduce the precomputation by a factor of 2.
		
%				\item<3-> Those two techniques can approximately shrink down the precomputation by a factor of 4.
	\end{itemize}
\end{block}

\begin{block}{Conclusion}
\small
\textbf{Conclusion:} $S_{n,r}$ over fixed points can be computed using at most $$\approx 2n\lceil h/2\rceil + q/(2\ell)$$
								additions, with the help of $\approx \ell n\lceil h/2 \rceil$ precomputed points
								\begin{equation*}
								\left\{mq^jP_i\ |\ 1\le i\le n,0\le j\le \lceil h/2\rceil-1,m\in \{1,2,...,2^{\ell-1}\}\right\},
								\end{equation*}	
			where $h = \lceil \log_q r\rceil$, and $q$ is a prime selected to minimize the complexity.
\end{block}

\setbeamercolor{block title}{fg=red,bg=white} % Change the block title color
\begin{block}{Acknowledgement}
This work is supported in part by xxx
\end{block}

\end{column} % End of the third column


\begin{column}{\sepwid}\end{column} % Empty spacer column

% The fourth column
\begin{column}{\onecolwid} 

\setbeamercolor{block title}{fg=blue,bg=white} 
\begin{block}{Acknowledgement}
This work is supported in part by xxx
\end{block}

\end{column} 
\end{columns} % End of all the columns in the poster

\end{frame} % End of the enclosing frame

\end{document}